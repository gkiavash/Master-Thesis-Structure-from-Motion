%! Author = ASUS
%! Date = 6/29/2023

% Preamble
\documentclass[11pt]{article}

% Packages
\usepackage{amsmath}

% Document
\begin{document}
    Structure from Motion is the task of calculating the 3D structure of a scene and the pose of cameras from
    a set of 2D images or video frames.
    It is a fundamental tool in many 3D Computer Vision applications such as 3D modeling, augmented reality,
    Robotics and Autonomous Systems.
    Several algorithms have been presented for this problem. However, in general, most of them follow the same procedure.
    The incremental Structure from Motion algorithm, \cite{schoenberger2016sfm}, consists of the following major steps.
    For every new image:
    Feature detection and Feature matching in 2D images,
    Camera pose estimation based on the matches,
    3D points extraction by triangulation,
    and refinement known as bundle adjustment.

    Since the algorithm is iterative, errors will be cascaded for every new image, and so, it may
    cause an exponential growth in the overall error. Sometimes, it fails and stops registering any new image,
    and the results converge to a complete wrong 3D scene. Therefore, accuracy and robustness of each part of
    the algorithm is crucial.

    Some of the key challenges in feature matching step, which is also a core task in many computer vision tasks,
    include occlusions, repetitive patterns, low-texture regions,or changes in lighting conditions.
    Developing robust feature matching techniques that can handle such challenges is an ongoing research focus.
    Noise and outliers, like moving objects, also can negatively impact the accuracy
    of the reconstructed 3D scene and camera poses. Moreover, SfM algorithms are computationally intensive,
    making them challenging to deploy in real-time or online applications where speed is crucial.

    Recent papers have shown great improvement in all these challenges. In this thesis, some of the best aproaches are
    reviewed and compared. Among these papers, "Pixel-Perfect Structure-from-Motion with Featuremetric Refinement",
    ~\cite{lindenberger2021pixsfm}, has shown a prominent performance. The proposed method in this paper generates a feature map per image using CNNs.
    Then, it adjusts the position of the existing keypoints, that are detected by any of methods mentioned in
    [\cite{wang2020displacement}, \cite{revaud2019r2d2}, \cite{detone2018superpoint}, \cite{dusmanu2019d2net}, and \cite{detone2018superpoint}]
    , by defining a flow from one point to another through a gradient descent update and a loss function based
    on the differences in their feature map values.
    Furthermore, in bundle adjustment's reprojection error, this paper considers the difference between the
    feature vectors instead of typical euclidean distance between the original 2D data points and the reprojected 3D points.
\end{document}