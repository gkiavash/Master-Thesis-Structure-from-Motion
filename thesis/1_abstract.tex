%! Author = gkiavash
%! Date = 09/01/2023

% Preamble
\documentclass[11pt]{article}

% Document
\begin{document}

3D data processing has become a prominent topic in Computer Vision nowadays.
One of the major areas of study in this field is Structure from motion (SfM)
which performs recovering the 3D structure of a scene and the pose of the
cameras from a set of 2D images taken from different viewpoints.

SfM algorithms can be used in a variety of applications, such as 3D modeling,
augmented reality, and autonomous driving.
Also, they can be used along with other computer vision techniques, like
object recognition and segmentation, in order to improve the analysis of the 3D structure.

Several different algorithms have been presented for this problem.
However, in general, most of them follow the same procedure.
The task of Structure from Motion consists of 5 general steps:
Feature detection and Feature matching, Camera pose estimation, Structure estimation, and Cumulative Error Reduction.

Recent papers have shown great refinement in all these steps.
In this paper, we will review these approaches, apply the SfM algorithm to an urban
dataset consisting of consecutive video frames, and then use the best paper to refine the result.
We will also investigate the effects of distortion and calibration of the camera in SfM's pipeline.

\end{document}
