%! Author = gkiavash
%! Date = 09/01/2023

% Preamble
\documentclass[11pt]{article}

% Document
\begin{document}
    \begin{abstract}
        Structure from Motion (SfM), the task of recovering 3D scene structure and camera poses from 2D images
        or video frames, is a prominent topic in 3D Computer Vision. SfM has applications in various areas
        such as 3D modeling, augmented reality, robotics, and autonomous systems. Recent research has made
        significant improvements in the accuracy and the challenges associated with SfM. This thesis reviews
        and compares state-of-the-art approaches with a special focus on
        "Pixel-Perfect Structure-from-Motion with Featuremetric Refinement" paper. In our experiment, several videos from
        the city of Padova were captured using a bike-mounted camera and processed through the SfM algorithm.
        The generated 3D reconstructions are refined and re-evaluated after applying the aforementioned method. Next, an algorithm is developed to register the generated local point clouds with a global, georeferenced point cloud of the whole city acquired by an airplane equipped with a high-resolution LiDAR. Qualitative and quantitative experiments demonstrate promising results in generating accurate 3D reconstruction and consistent alignments between the reconstructed local point clouds and the global point cloud.
    \end{abstract}

\end{document}
