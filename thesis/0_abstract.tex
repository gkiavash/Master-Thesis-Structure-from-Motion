%! Author = gkiavash
%! Date = 09/01/2023

% Preamble
\documentclass[11pt]{article}

% Document
\begin{document}
    \begin{abstract}
        Structure from Motion (SfM), the task of recovering 3D scene structure and camera poses from 2D images
        or video frames, is a prominent topic in 3D Computer Vision. SfM has applications in various areas
        such as 3D modeling, augmented reality, robotics, and autonomous systems. Recent researches have made
        significant improvements in the accuracy and the challenges associated with SfM. This thesis reviews
        and compares state-of-the-art approaches with an especial focus on
        "Pixel-Perfect Structure-from-Motion with Featuremetric Refinement" paper. In our experiment, several videos from
        the city of Padova were captured using a bike-mounted camera and processed through the SfM algorithm.
        The generated point clouds are refined and re-evaluated by the aforementioned paper. Next, an algorithm is
        developed to localize the generated point clouds in an extensive point cloud dataset of the whole city,
        which had been acquired by an airplane. This thesis demonstrates promising results in localization by generating accurate
        and sufficient 3D points from streets using images and video frames.
    \end{abstract}

\end{document}
