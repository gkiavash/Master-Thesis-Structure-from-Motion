% Preamble
\documentclass[11pt]{article}

% Packages
\usepackage{amsmath}

% Document
\begin{document}
    \section{Structure from Motion Revisited}
    Structure from Motion is the task of calculating the 3D structure of a scene and the pose of cameras from
    images captured in multiple views or video frames. SfM is an important tool in many applications like
    Visual SLAM, 3D scanning, and Autonomous Driving. There are several algorithms used in SFM that help
    in reconstructing the 3D structure. In this chapter, Incremental Structure from Motion will be explained
    as it is the base algorithm for most papers in these criteria. The algorithm can be viewed as a pipeline 
    that contains the following separated steps:

    \paragraph{1. Feature detection and matching:} The first step in SFM is feature detection and matching, where
    the algorithm identifies common features or points in the 2D images or video frames. These features
    can be edges, corners, blobs, or other distinctive patterns that can be detected consistently across
    the images. The algorithm, then, matches these features between different input frames. This step is one
    of the most challenging parts in SfM. Occlusion, illumination changes, perspective transformation, and moving objects
    are some of them. However, since this is a core task in many Computer Vision tasks, there has been
    tons of researches in feature detection and matching, including deep based approaches. SIFT, SuperPoint, 
    S2-Net, R2D2 are often used among SfM related papers.
    
    \paragraph{2. Geometric Verification:} raw matches are not enough for the scene reconstruction.
    Matches in similar repeated patterns, multiple objects in one image, etc, are some issues that may cause wrong
    poses and 3D points. Outliers must be removed from the list of matches. Therefore, by using RANSAC and epipolar
    geometry, camera matrices, i.e. homography(H), Essential(E), and Fundamental(F) matrices, can be calculated
    to verify the geometric consistency of the reconstructed 3D points and camera poses.
    Without geometric verification, the reconstructed 3D model may contain errors that can lead to incorrect result.

    \paragraph{2. Reconstruction Initialization:} If the pipeline is considered as an iterative or sequential process,
    a starting point is essential. Choosing a good initial pair of images is, also, so important. At the end of
    each iteration, the camera parameters and 3D structure are optimized, i.e. the differences between
    the observed 2D features and the projected 3D points are minimized. Without a good initial estimate,
    the optimization process usually doesn't converge and stuck in local minima. A good initial pair can be
    chosen from the top of list of images that are sorted based on the number of correspondences matches.

    \paragraph{3. Image Registration and Triangulation:} After initialization, the algorithm uses triangulation to estimate
    the 3D position of the matched features in every new image. Triangulation involves finding the intersection point
    of two or more rays of light that originate from the camera and pass through the 2D features. The intersection
    point represents the 3D position of the feature in the scene.

    \paragraph{4. Bundle Adjustment:} As the previous processes involve some errors, there may be inconsistencies
    or misalignment between the estimated camera parameters and the triangulated 3D points, and it could get worse
    and worse for every new image. Bundle adjustment refines those parameters by rewriting the non-linear
    reprojection equations which contain camera poses' matrices, calibration parameters, and, 3D points positions.
    It typically uses an optimization algorithm such as Levenberg-Marquardt to minimize
    the reprojection error. The optimization problem can be non-linear and high-dimensional, and it may take a
    long time to converge to a good solution. However, bundle adjustment can significantly improve the accuracy
    of the reconstructed 3D scene and camera poses.

    \section{Related Works}
    Structure of Motion is a core task for many projects. It gives the ability to obtain detailed 3D models
    of the environment from 2D images or video sequences, and these information can be used widely in object tracking,
    object detection, and many other low level tasks, and also, understanding the scene and detecting speed and
    positions of objects for autonomous driving as a high level task. Therefore, the accuracy and the speed
    of the algorithm is crucial. As an example for accuracy, a tiny error, e.g. 1 degree, in the rotation of
    a camera could cause several meters misalignment in the coordinates of the points especially in open scenes.
    So, over the past few decades, there has been lots of efforts to optimize each part of these algorithms.
    Here is a review of the recent best papers in this area:

    \newpage
    \section{Researches in Feature Matching}
    Each paper usually focuses on one or two tasks in the SfM pipeline. The first step, feature matching,
    is one of the biggest challenges in Computer Vision as they are used to identify the corresponding
    matches among the input images. Because of that, there has been tons of reasearches in this area.
    Some of the chanllenges are moving objects, deformation in different viewpoints, occlusion, illumination
    changes, etc.

    ~\cite{Dusmanu2020Multi} tracks each keypoint in all input images, and creates a tentative
    matches graph with keypoints as nodes and matches as edges. Each keypoint is known by its surrounding pixels
    information. So, a h*w patch is calculated by CNN. Then, each pair of patches in the graph is refined(or better
    to say aligned) by a neural network with Siamese architecture followed by a correlation layer as dot product
    similarity. Each keypoint could be seen in multiple images. A single incorrect match or displament can cause
    cascade of errors in the results. Therefore, another refinement is also applied to track of each keypoint
    detected in multiple images. For each track, a non-linear equation for minimizing dot similarity between
    patches is optimized. The idea of this paper is also used in \cite{lindenberger2021pixsfm} which will be
    described in detail in the next chapter.

    ~\cite{wang2021deep} uses optical flow to predict dense correspondences between two frames. The method is
    able to find better matching points and therefore more accurate poses and depth maps, especially for
    textureless and occluded areas. They use DICLFlow, explained in \cite{wang2020displacement} to generate
    dense matching points between two consecutive frames. This method uses a displacement invariant matching
    cost learning strategy and a soft-argmin projection layer to ensure that the network learns dense matching
    points rather than image-flow regression. Noisiy matches like moving
    objects are filtered out by the means of SIFT features. After calclulating the essential matrix , using
    the classic five-point algorithm with RANSAC scheme, with the dense matching points from optical flow estimation,
    the dense depth map is computed by performing triangulation. Before this process, the matching is performed
    again by constraining the search space to epipolar lines computed from the relative camera poses. They also,
    introduced a Scale-Invariant Depth Module to deal with the up-to-scale relative pose and mismatch between the
    scale of the camera motion and the scale of the depth map. #TODO.

    \newpage
    \section{Researches in Bundle Adjustment}
    Bundle Adjustment algorithms optimizes the camera parameters and the output points by minimizing the
    difference between the observed image points and the corresponding projected 3D points. However, there
    are some disadvantages with this technique. Bundle Adjustment requires solving a large system of
    non-linear equations, which can be computationally expensive, especially for large datasets. This can be a
    significant drawback, especially when processing real-time data. It can be sensitive to the initial values.
    If the initial values are not accurate, the optimization may fail converge. The optimization problem can have
    many local minima which means high probability in resulting in suboptimal or incorrect results.



    ~\cite{tang2019banet} mainly focuses on bundle adjustment step in SfM pipeline. They implemented feature-metric
    bundle adjustment that minimizes feature-metric errors between feature pyramid of images obtained by CNNs.
    As \cite{LSDSLAM} says, there are two types of BAs:
    - Typical geometric BA with re-projection error(pixel coordinates): Only a few pixels, i.e. keypoints, are
    taken into account which comes with keypoints detection and matching challenges
    - Photometric BA algorithm(all aligned pixel, pixel intensities as error): It has good accuracy,
    especially at less textured scenes. However, the disadvantages would be sensitivity to camera exposure,
    illumination changes, and outliers such as moving objects. Also, considering all pixels would increase
    the computation dramatically.
    BA-Net creates a pyramid of features for each image and align them. The featire pyramid is generated by
    multi-scale hierarchy of CNN(DRM-54, \cite{yu2017dilated}) to construct feature pyramids. Then, the BA equation
    to minimize would be:
    \[ e^f_{i,j}(X) &= F_{i}(\pi(T_{i},d_{j} \cdot q_{j}))-F_{1}(q_{j}) \]
    where F = {Fi | i = 1  Ni} are feature pyramids of images I = {Ii | i = 1 · · · Ni}

    For the generation of dense reconstruction, the depth of all pixels in all images are required. However, the
    computation of these values is super expensive. This paper, also, uses b a great approach to deal with this
    issue that is mentioned in \cite{tateno2017cnnslam} and \cite{yang2018deep}. A set of arbitrary basis depth maps
    is created. Then, the final depth map is generated as a linear combination of these basis depth maps, which
    is: D = ReLU(w^T * B). "w" will be optimized in our BA-Layer. Generally, this idea is great when the number of
    values to calculate is too high.

    \newpage
    \section{Pixel-Perfect Structure-from-Motion with Featuremetric Refinement}
    Among all reviewed paper, \cite{lindenberger2021pixsfm} has the best performance. The paper
    proposes two stages to improve the accuracy of structure-from-motion for 3D reconstruction.
    In the first stage, the initial keypoint locations are adjusted prior to any geometric estimation
    by optimizing a direct cost over dense feature maps. In the second stage, 3D points and camera poses
    are refined as post-processing using a featuremetric error based on dense features map obtained by
    Convolutional Neural Network.
    The paper produced accurate reconstructions and scaled well to large scenes with thousands of images.
    Also, their contribution could be used as an extra step in the SfM pipeline. So, it can be easily
    integrated with any other papers' contributions.

    They use the initial idea of \cite{Dusmanu2020Multi} that separates the tracks for each keypoint, and
    adjusts their locations by optimizing the geometric cost over the input images which the keypoint
    is seen. \cite{Dusmanu2020Multi} improved SfM, but has limited accuracy and scalability. To solve that, This paper proposes
    a feature map for each image using convolution neural networks. The feature map condenses the dense
    information from surrounding pixels into a vector for each pixel. The reason behind this step is the
    fact that refining geometry is an inherently local operation, and the dense information only needs to be
    locally accurate and invariant but not globally discriminative.
    For the first stage, i.e. keypoint adjustment refinement, the locations of 2D keypoints belonging to the same
    track j is refined by optimizing its featuremetric consistency along tentative matches. The loss function is

    \[ E_{FKA}^j &= \sum_{(u,v) \in M_{j}}w_{uv}\lVert F_{i(u)}[p_{u}] - F_{i(v)}[p_{v}]\rVert_{\gamma} \]


    In compare to \cite{Dusmanu2020Multi}, optimizing the cost of feature maps instead of the patch neighboring
    pixels has better efficiency.

    In the second stage, i.e. bundle adjustment refinement, typical approach uses the euclidean distance between the point and
    its reprojected 3D from another view. However, this paper considers the difference between the feature vectors
    of the point and the feature vector of the point where the 3D point is reprojected. The loss fubnction is

    \[ E_{FBA} &= \sum_{j} \sum_{(i,u) \in \tau_{j}}\lVert F_{i}[ \prod (R_{i}P_{j} + t_{i},C_{i})] - f^j\rVert_{\gamma} \]


    Also, instead of acquiring the cost equation for each pair of views, one image is considered as a reference,
    and the cost equations are written between the reference image and the rest of views. it reduces the number
    of residuals from O(N^2) to O(N). This is not good for the sequential video frames. Because newer frames will be
    too far from the reference image. However, it is achievable if the new input is a batch of new images, and
    the refinement is applied to each batch separately.

    As it is







\end{document}
